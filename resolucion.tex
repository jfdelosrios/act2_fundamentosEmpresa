\documentclass{article}  % Tipo de documento
\usepackage[a4paper, left=2cm, right=2cm, top=2.5cm, bottom=2.5cm]{geometry}

\usepackage[utf8]{inputenc}  % Codificación del archivo
\usepackage{hyperref}
\usepackage{longtable} % Paquete para tablas largas
\usepackage[spanish]{babel}

\title{Análisis PESTEL y Viabilidad de Mercadona en Marruecos}  % Título del documento

\author{
Matias Agudelo\and
Jorge Fernández de los Ríos\and
Oscar Ivan Ramirez\and
Carlos Guillermo Tapia
\\
\\
Universidad Internacional De la Rioja\\
 Fundamentos de Empresa
}  % Autores

\date{\today}  % Fecha 
\begin{document}  % Comienzo del documento

\maketitle  % Crear el título

\section{Introducción.} % Sección 1

El presente informe tiene como objetivo realizar un análisis PESTEL para evaluar la viabilidad de la entrada de la empresa española Mercadona en el mercado marroquí. Se considerarán factores de dimensiones políticas, económicas, socioculturales, tecnológicas, ecológicas, ambientales y legales que podrían afectar a la operación de Mercadona en Marruecos.

\section{Elección de los factores relevantes para la empresa.}

A continuación, se definen los límites del análisis seleccionando cuatro factores relevantes para cada dimensión. Estos factores se extraen de la figura 1 del archivo de la actividad (gcd08\_act2.docx).

\begin{longtable}{|c|l|}
\caption{Factores Relevantes por Dimensión} \label{tab:factores_relevantes} \\

\hline
\textbf{Dimensión} & \textbf{Factores Relevantes} \\
\hline
\endfirsthead

\hline
\multicolumn{2}{|c|}{\textbf{Continúa la tabla}} \\
\hline
\textbf{Dimensión} & \textbf{Factores Relevantes} \\
\hline
\endhead

\hline
\endfoot

\hline
Política & - Estabilidad política. \\ 
         & - Comercio exterior con España. \\ 
         & - Política fiscal de contención. \\ 
         & - Derechos humanos y libertad de expresión. \\ 
\hline
Economía & - Crecimiento económico. \\ 
         & - Inflación. \\ 
         & - Desempleo. \\ 
         & - Déficit fiscal. \\ 
\hline
Sociocultural & - Tamaño de la población. \\ 
              & - Poder adquisitivo. \\ 
              & - Barreras culturales y religiosas. \\ 
              & - Nivel educativo. \\ 
\hline
Tecnología & - Penetración de internet. \\ 
           & - Digitalización de la economía. \\ 
           & - Iniciativas de emprendimiento tecnológico. \\ 
           & - Brecha digital. \\ 
\hline
Ecológica y ambiental & - Políticas de sostenibilidad. \\ 
                       & - Cambio climático. \\ 
                       & - Energías renovables. \\ 
                       & - Gestión de residuos. \\ 
\hline
Legal & - Legislación laboral. \\ 
      & - Libertad de expresión. \\ 
      & - Protección a la inversión extranjera. \\ 
      & - Legislación comercial UE-Marruecos. \\ 
\hline
\end{longtable}


\section{Cuantificación de los factores elegidos de cada dimensión.}

A continuación se valorará cada uno de los factores clave en una escala que de 1 a 5, donde 1 es muy negativo y 5 muy positivo.

\begin{longtable}{|l|l|c|}
\caption{Factores Relevantes por Dimensión y su Valoración} \label{tab:factores_valoracion} \\

\hline
\textbf{Dimensión} & \textbf{Factor} & \textbf{Valoración (1-5)} \\
\hline
\endfirsthead

\hline
\multicolumn{3}{|c|}{\textbf{Factores Relevantes por Dimensión y su Valoración (continuación)}} \\
\hline
\textbf{Dimensión} & \textbf{Factor} & \textbf{Valoración (1-5)} \\
\hline
\endhead

\hline
\endfoot

\hline
Política & Estabilidad política         & 4 \\
 & Comercio exterior con España & 5 \\
 & Política fiscal de contención & 4 \\
 & Derechos humanos y libertad de expresión & 2 \\
\hline
Economía & Crecimiento económico        & 3 \\
 & Inflación                    & 2 \\
 & Desempleo                    & 3 \\
 & Déficit fiscal               & 3 \\
\hline
Sociocultural & Tamaño de la población       & 4 \\
 & Poder adquisitivo            & 3 \\
 & Barreras culturales y religiosas & 3 \\
 & Nivel educativo              & 3 \\
\hline
Tecnología & Penetración de internet      & 4 \\
 & Digitalización de la economía & 4 \\
 & Iniciativas de emprendimiento tecnológico & 4 \\
 & Brecha digital               & 2 \\
\hline
Ecológica y ambiental & Políticas de sostenibilidad & 4 \\
 & Cambio climático             & 3 \\
 & Energías renovables          & 4 \\
 & Gestión de residuos          & 3 \\
\hline
Legal & Legislación laboral          & 3 \\
 & Libertad de expresión        & 2 \\
 & Protección a la inversión extranjera & 4 \\
 & Legislación comercial UE-Marruecos & 4 \\
\hline
\end{longtable}


\section{Análisis Detallado de
Factores.}\label{anuxe1lisis-detallado-de-factores.}

\subsection{Dimensión política.}\label{dimensiuxf3n-poluxedtica.}

\textbf{Estabilidad política:} La estabilidad del sistema político en
Marruecos proporciona un entorno relativamente seguro para las
inversiones extranjeras, aunque persisten riesgos en materia de derechos
civiles.

\textbf{Comercio exterior con España:} España es el primer socio
comercial de Marruecos, facilitando un entorno favorable para empresas
españolas como Mercadona.

\textbf{Política fiscal de contención:} El compromiso del gobierno
marroquí con la estabilidad fiscal reduce riesgos macroeconómicos para
los nuevos entrantes.

\textbf{Derechos humanos y libertad de expresión:} Limitaciones en
libertades básicas podrían representar riesgos reputacionales y de
imagen para empresas extranjeras .

\subsection{Dimensión economía.}\label{dimensiuxf3n-economuxeda.}

\textbf{Crecimiento económico:} Aunque positivo, el crecimiento marroquí
es moderado y depende en parte de sectores vulnerables como la
agricultura.

\textbf{Inflación:} El alza de precios podría afectar el poder
adquisitivo de los consumidores locales, impactando la demanda
minorista.

\textbf{Desempleo:} Una tasa de desempleo alta sugiere una demanda
interna limitada en algunos segmentos, aunque ofrece disponibilidad
laboral.

\textbf{Déficit fiscal:} El déficit controlado genera un ambiente de
estabilidad financiera relativa para los nuevos negocios.

\subsection{Dimensión sociocultural.}\label{dimensiuxf3n-sociocultural.}

\textbf{Tamaño de la población:} Una población de más de 36 millones
crea un mercado potencial significativo para bienes de consumo como los
de Mercadona.

\textbf{Poder adquisitivo:} Aunque bajo comparado con Europa, el
crecimiento de la clase media ofrece oportunidades de expansión
controlada.

\textbf{Barreras culturales y religiosas:} La fuerte influencia del
islam y las tradiciones puede requerir adaptación en la oferta de
productos alimentarios.

\textbf{Nivel educativo:} Mejoras en educación favorecen una población
laboral más capacitada, aunque persisten retos en zonas rurales.

\subsection{Dimensión tecnología.}\label{dimensiuxf3n-tecnologuxeda.}

\textbf{Penetración de internet:} El acceso generalizado a internet
permite estrategias de marketing digital y operaciones de e-commerce.

\textbf{Digitalización de la economía:} La transformación digital
facilita la adopción de tecnologías de eficiencia logística y ventas
multicanal.

\textbf{Iniciativas de emprendimiento tecnológico:} El apoyo al
emprendimiento fomenta un entorno dinámico donde Mercadona puede aliarse
con innovadores locales.

\textbf{Brecha digital:} Persisten desigualdades tecnológicas entre
zonas urbanas y rurales que podrían limitar el acceso a ciertos
segmentos de consumidores.

\subsection{Ecológica y ambiental.}\label{ecoluxf3gica-y-ambiental.}

\textbf{Políticas de sostenibilidad:} Marruecos impulsa políticas verdes
que favorecen estrategias de responsabilidad social empresarial.

\textbf{Cambio climático:} Fenómenos climáticos extremos afectan la
producción agrícola y logística, representando riesgos operativos.

\textbf{Energías renovables:} El impulso a renovables puede traducirse
en oportunidades para una cadena de suministro más sostenible.

\textbf{Gestión de residuos:} Limitaciones en infraestructura de
residuos implican la necesidad de políticas internas de manejo de
desperdicios.

\subsection{Dimensión legal.}\label{dimensiuxf3n-legal.}

\textbf{Legislación laboral:} Existen leyes laborales razonablemente
estructuradas, pero su aplicación en la práctica puede ser deficiente.

\textbf{Libertad de expresión:} La represión de libertades básicas
representa un riesgo para la reputación de marcas occidentales.

\textbf{Protección a la inversión extranjera:} Los acuerdos
internacionales y la liberalización de mercados brindan un marco
atractivo para invertir.

\textbf{Legislación comercial UE-Marruecos:} Los acuerdos comerciales
facilitan la importación y exportación de productos entre ambos bloques.

\section{Informe Final: Viabilidad de Mercadona en
Marruecos.}\label{informe-final-viabilidad-de-mercadona-en-marruecos.}

La expansión de Mercadona en Marruecos presenta un conjunto de
oportunidades y retos que deben ser cuidadosamente gestionados. Desde el
punto de vista político, la estabilidad y los fuertes lazos comerciales
con España son aspectos positivos, aunque existe un riesgo reputacional
asociado a las limitaciones de libertades civiles.

Económicamente, el crecimiento moderado, el control fiscal y la
estabilidad monetaria proporcionan un entorno razonablemente seguro, si
bien la inflación y el desempleo reducen el poder adquisitivo de algunos
segmentos del mercado.

El contexto sociocultural muestra un mercado amplio pero con barreras
culturales y desigualdades que Mercadona deberá gestionar mediante
estrategias de localización de productos. Tecnológicamente, el país
ofrece un entorno favorable para la expansión digital, aunque persisten
brechas de acceso en zonas rurales.

Desde una perspectiva ecológica y legal, Marruecos se alinea
progresivamente con las tendencias internacionales, apoyando energías
renovables y protegiendo inversiones extranjeras, aunque siguen
existiendo riesgos derivados de la limitada protección de ciertos
derechos civiles.

\textbf{Recomendaciones estratégicas:}


\begin{itemize}
\item
  Adaptar la oferta de productos alimenticios respetando costumbres
  islámicas (por ejemplo, productos halal).
\item
  Implementar estrategias de precios asequibles para captar a la
  creciente clase media.
\item
  Fortalecer la estrategia omnicanal combinando tiendas físicas y
  canales digitales para maximizar cobertura.
\end{itemize}



\section{Conclusión:}\label{conclusiuxf3n}

La viabilidad de Mercadona en Marruecos es positiva, siempre y cuando la
empresa adopte estrategias adaptativas, sociales y ecológicas en su
implantación.

\section{Bibliografía.}\label{bibliografuxeda.}


\begin{itemize}
\item
  Ministerio de Asuntos Exteriores, Unión Europea y Cooperación de
  España. (2024). MARRUECOS\_FICHA PAIS.pdf.
\item
  World Bank. (2024). DataBank - Marruecos. Retrieved from
  \url{https://databank.worldbank.org/}
\item
  The Economist Intelligence Unit. (2024). Morocco Economy, Politics and
  GDP Growth Summary.
\item
  Human Rights Watch. (2024). Informe Mundial 2024: Marruecos y el
  Sáhara Occidental.
\item
  Fondo Monetario Internacional. (2023). Perspectivas de la Economía
  Mundial.
\item
  Maroc.ma. (2024). La estrategia nacional de la transición digital de
  Marruecos.
\end{itemize}


\end{document}  % Fin del documento
